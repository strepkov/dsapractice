\chapter{Evaluation}

% What are the results?
% Whats works good?
At the end of the workshop, we were able to have our algorithm idea implemented. Due to the timeframe of the workshop, we were only able to manually test the essential functions, i.e. theft detection via the accelerometer and suspicious GPS movements. As soon as the device is moved (e.g. lifted up from a table), the accelerometer-based detection correctly fired. Additionally, we field-tested the GPS-based detection by running down the street to simulate a noticeable change in location. To be sure, that we do not tested the accelerometer again, we put the the box on the ground and waited a few seconds until the buffer of the accelerometer proxy was reset and only the GPS-based detection could be able to fire a theft signal. 

However, in the scope of the workshop, much of the FMS data was not available. Thus, not all branches from our decision tree could be tested. This could have been achieved by simulating the FMS data. However, we did not have the necessary time left towards the end of the workshop. We could check via FMS, if a trailer is connected to an truck, if the wheels are spinning or if the engine is running, but we could not consider the other branches of our decision tree, because it was not possible to connect the box to an CAN-Bus and generate test data for a running engine or wheel speed information. 

% Where are weaknesses in the algorithm? (Stolen vehicle with key)
Theft detection is a challenging problem. Even though all the different branches in our decision cover many possible ways to steal a vehicle, there are cases which are not covered -- and cases that are hard to detect in general. This includes the case where a thief has gained access to the vehicle keys, thus leaving no way to detect a theft in non-commercial vehicles (which do not require a driver card). This is because we can not say, if the truck is stolen or not. Technically it looks like a normal usage of the truck, because the wheels are spinning and the engine is on during the movement.


\clearpage